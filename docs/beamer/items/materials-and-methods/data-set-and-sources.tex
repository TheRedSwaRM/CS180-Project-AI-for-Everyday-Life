\subsection{Data set and sources}
\begin{frame}
\frametitle{\subsecname}
	\begin{itemize}
		\item As previously stated in the project background, the data set used for the implementation of the project is the \textit{Students' Employability Dataset - Philippines} readily available from \textit{Kaggle}.
		\item \url{https://www.kaggle.com/datasets/anashamoutni/students-employability-dataset}
		\item Owned by \textit{Anas Hamoutni}.
	\end{itemize}
\end{frame}

\begin{frame}
\frametitle{\subsecname}
	\begin{itemize}
		\item Notes about the data set (as shown from Kaggle):
		\begin{itemize}
			\item The data set was collected from different university agencies in the Philippines
			\item The data set consists of mock job interview results of 2,982 observations.
			\item However, the dataset collected requires normalization and cleaning.
			\item The data set that was collected is compliant with the Data Privacy Act of the Philippines.
		\end{itemize}
	\end{itemize}
\end{frame}

\begin{frame}[fragile]
\frametitle{Data set format}
	\begin{itemize}
		\item The data set is in \texttt{.xlsx} (MS Excel Spreadsheet) format.
		\item Can be imported using \texttt{read\_excel} from \texttt{pandas}.
\begin{minted}{python}
data = pd.read_excel('../data/Student-Employability-Datasets.xlsx')
\end{minted}
	\end{itemize}
\end{frame}

\begin{frame}
\frametitle{Data set as shown in MS Excel}
\wfigo{dataset}
\end{frame}

\begin{frame}
\frametitle{Data set as shown in Python Notebook}
\wfigo{pynb-dataset}
\end{frame}
