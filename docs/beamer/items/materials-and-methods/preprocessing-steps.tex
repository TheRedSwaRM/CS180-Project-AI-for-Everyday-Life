\subsection{Preprocessing steps}
\begin{frame}
\frametitle{\subsecname}
	\begin{itemize}
		\item Before performing the MLP algorithm into our data set, we should consider normalizing and cleaning the data set first to achieve best results.
	\end{itemize}
\end{frame}

\begin{frame}[fragile]
\frametitle{Data cleaning}
	\begin{itemize}
		\item Since the data set is well-structured from the start (i.e., no \texttt{null} entries), all we need to do is to remove the \texttt{Name of Student} and \texttt{CLASS} columns in our data set prior to splitting the data set into \texttt{train} and \texttt{test} sets. 
\begin{minted}{python}
data = data.drop("Name of Student", axis=1)
x = data.drop('CLASS', axis = 1)
\end{minted}
	\end{itemize}
\end{frame}

\begin{frame}[fragile]
\frametitle{Data normalization}
	\begin{itemize}
		\item Since we are using the MLP algorithm, we use \texttt{MinMaxScaler} with \texttt{tanh} activation to normalize our data set.
\begin{minted}{python}
#Normalize train and test sets for MLP
from sklearn.preprocessing import MinMaxScaler

#TANH activation
scaler = MinMaxScaler(feature_range=(-1,1))
\end{minted}
		\item We will use \texttt{scaler} later on the transformation of our \texttt{train} and \texttt{test} sets.
	\end{itemize}
\end{frame}
